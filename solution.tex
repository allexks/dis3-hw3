\documentclass{article}

\usepackage{amsfonts}
\usepackage{amsmath}
\usepackage[utf8]{inputenc}
\usepackage[bulgarian]{babel}

\title{Курсова работа №3}
\author{Александър Игнатов \\ Ф№ 62136 }
\date{\today}


\begin{document}

\maketitle

\section*{Задача}

Като направите подходящо развитие в степенен ред на подинтегралната функция пресметнете с точност \( E = 10^{-4} \) определения интеграл:

\[
    \int_0^\frac{1}{4}e^{-x^2}dx
\]

\section*{Решение}

Редът на Маклорен за експоненциалната функция \( e^x \) е следната сума:

\[
    \sum_{n=0}^{\infty} \frac{x^n}{n!}
\]

Като заместим \( x \) с \( -x^2 \) получаваме:

\begin{align*}
    & \int_0^\frac{1}{4}e^{-x^2}dx = \\
    &= \int_{0}^{\frac{1}{4}} \sum_{n=1}^{\infty} \frac{(-x^2)^{n}}{n!} \, dx = \\
    &= \sum_{n=1}^{\infty} \frac{(-1)^n}{n!} \int_{0}^{\frac{1}{4}} x^{2n} \, dx = \\
    &= \sum_{n=0}^{\infty} \frac{(-1)^nx^{2n+1}}{n!(2n+1)}\biggr\rvert_0^\frac{1}{4} = \\
    &= \sum_{n=0}^{\infty} \frac{(-1)^n \left(\frac{1}{4}\right)^{2n+1}}{n!(2n+1)} - 0 = \\
    &= \sum_{n=0}^{\infty} \frac{(-1)^n}{4^{2n+1}n!(2n+1)}
\end{align*}

Нека \( k \in \mathbb{N}_0 \):

\[
    \sum_{n=0}^{\infty} \frac{(-1)^n}{4^{2n+1}n!(2n+1)} = \sum_{n=0}^{k} \frac{(-1)^n}{4^{2n+1}n!(2n+1)} + \sum_{n=k+1}^{\infty} \frac{(-1)^n}{4^{2n+1}n!(2n+1)}
\]

Нека означим с \( R_k \) втория член на сбора (този, при който \( n \) пробягва от \( k + 1\) до \( \infty \)).

Тогава имаме, че

\begin{gather}
    R_k > \frac{1}{4^{2k+3}(k+1)!(2k+3)}
\end{gather}


По условие искаме да пресметнем израза с точност \( E = 10^{-4} \), т.е.

\begin{gather}
    |R_k| < 10^{-4}
\end{gather}

От (1) и (2) извеждаме неравенството

\begin{gather*}
    4^{2k+3}(k+1)!(2k+3) > 10^4
\end{gather*}

Минималната стойност на \( k \), за която то е изпълнено, е \( k = 1 \).

Така пресмятаме даденият в условието израз:

\[
    \int_0^\frac{1}{4}e^{-x^2}dx = \frac{1}{4} - \frac{1}{192} + R_1 \approx 0,2448
\]


\end{document}

